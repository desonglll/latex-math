\chapter{剑指150}
开始于2023年3月1日\\高昆轮
\section{第一章 函数极限连续}
\subsection{考点1 常用函数及常用曲线p4}
\subsubsection{函数的概念及一些常用函数}
\begin{enumerate}
    \item 绝对值函数
    y=|x|在x=0处连续,但是是不可导的(左导数不等于右导数)
    $|a_1 \pm a_2 \pm... \pm a_n|\le|a_1|+|a_2|+...+|a_n|$
    \item 最值函数\\
    $U=max\{f(x),g(x)\}= \frac{f(x)+g(x)+|f(x)-g(x)|}{2}$\\
    $V=min\{f(x),g(x)\}= \frac{f(x)+g(x)-|f(x)-g(x)|}{2}$\\
    $U+V=f(x)+g(x),U-V=|f(x)-g(x)|,U \bullet V=f(x)g(x)$
    \item 符号函数\\
    $对任何x,都有\sqrt{x^2}=|x|=xsgn(x)$
    \item 取整函数\\
    $[x+n]=[x]+n,其中n是整数,x-1<[x]\le x $
    \item 幂指函数\\
   $ u(x)^{v(x)}=e^{v(x)lnu(x)}$
    \item 狄利克雷函数
    \item 
    $有理数无理数$
\end{enumerate}


\section{考点2 函数的几种特性(特别要记忆对这些特性总结的结论)p11}

\section{考点3 极限的定义p14}

\section{考点4 极限的性质p19}

\section{考点5 无穷小与无穷大p20}

\section{考点6 极限的四则运算法则及两个重要极限p25}

\subsection{极限的四则运算法则}

\begin{enumerate}
    \item 若$\lim f(x)=A (\exists)$,$\lim g(x)=B (\exists)$
    \begin{enumerate}
        \item $\lim [f(x)\pm g(x)]=\lim f(x)\pm g(x)=A\pm B$
    \end{enumerate}
    \item 若$\lim f(x)$存在,$\lim g(x)$不存在,则$\lim [f(x)\pm g(x)]$必不存在
    \item 若$\lim f(x)$不存在,$\lim g(x)$不存在,则$\lim [f(x)\pm g(x)]$不一定不存在(可能存在)
    \item 若$\lim \frac{f(x)}{g(x)}=A$,且$\lim g(x)=0$,则$\lim f(x)=0$ “母为0,推子为0”
    \item 若$\lim \frac{f(x)}{g(x)}=A\ne 0$,且$\lim f(x)=0$,则$\lim g(x)=0$ “子为0,推母为0”
\end{enumerate}

\section{考点7 等价代换p29}

\section{考点8 洛必达法则}

\section{考点9 泰勒公式(多项式的高次逼近)p35}

\subsection{泰勒公式}
\newpage
\subsubsection{泰勒公式表}
% \begin{multicols}{2}
\begin{itemize}
\item $ \sin x=x-\frac{x^{3}}{3!}+o(x^{3}) $
\item $ e^{x}=1+x+\frac{x^{2}}{2!}+\frac{x^{3}}{3!}+o(x^{3}) $
\item $ \arcsin x=x+\frac{x^{3}}{3!}+o(x^{3}) $
\item $ \tan x=x+\frac{x^{3}}{3}+o(x^{3}) $
\item $ \cos x=1-\frac{x^{2}}{2!}+\frac{x^{4}}{4!}+o(x^{4}) $
\item $ ln(1+x)=x-\frac{x^{2}}{2}+\frac{x^{3}}{3}+o(x^{3}) $
\item $ \arctan x=x-\frac{x^{3}}{3}+o(x^{3}) $
\item $ (1+x)^{\alpha}=1+\alpha x+\frac{\alpha (\alpha -1)}{2!}x^{2}+\frac{\alpha (\alpha -1)(\alpha -2)}{3!}x^{3}+o(x^{3}) $
\item $ ln(x+\sqrt{1+x^2})=x-\frac{1}{6}x^3+\frac{3}{40}x^5+o(x^3) $
\end{itemize}
\begin{figure}
    \centering %表示居中
    % \includegraphics[height=0.5\paperheight]{./img/taylor-expansions.JPG}
\end{figure}


% \end{multicols}

\subsubsection{用泰勒公式求极限}
\begin{enumerate}
\item $ \frac{A}{B} $: 适用于``上下同阶''的原则 \par
如果分母(或者分子)是$ x $的$ k $此幂, 则应该把分子(或分母)展开到$ x $的$ k $次幂.
\item $ A-B $: 适用于幂次最低原则 \par
将$ A, B $分别展开到它们的系数不相等的$ x $的最低次幂为止.
\end{enumerate}

\section{考点10 幂指函数$u(x)^{v(x)}$的极限p38}

\section{考点11 已知极限反求参数及无穷小阶数的比较p40}

\section{考点12 数列的极限p43}

\section{考点13 函数的连续性与间断点p48}

\section{考点14 闭区间上连续函数的性质p53}

\chapter{一元函数微分学}

\section{考点15 导数定义}

\section{考点16 四则、复合函数、反函数及对数求导法则p62}

\section{考点17 高阶导数p65}

\section{考点18 隐函数及由参数方程所确定的函数的求导法则p67}

\section{考点19 分段函数及绝对值函数求导p70}

\section{考点20 导数的几何、物理意义及相关变化率p75}

\section{考点21 函数的微分p77}

\section{考点22 中值定理p80}

\section{考点23 单调性与极值、最值p91}

\section{考点24 凹凸性与拐点p99}

\section{考点25 渐近线、曲率p102}

\section{考点26 函数图形的描绘p105}

\section{考点27 证明函数或常数不等式p106}

\section{考点28 用导数讨论方程的根p109}

\chapter{一元函数积分学}

\section{考点29 原函数与不定积分的概念、积分公式p114}

\section{考点30 凑微分法求不定积分p117}

\section{考点31 换元法求不定积分p120}

\section{考点32 分布积分法求不定积分p121}

\section{考点33 有理函数的积分p124}

\section{考点34 定积分的定义及性质p125}

\section{考点35 定积分的计算方法及若干技巧p132}

\section{考点36 变限积分函数及其求导原理p140}

\section{考点37 变限积分函数的综合题p145}

\section{考点38 定积分不等式的证明p150}

\section{考点39 定积分与极限的综合题p154}

\section{考点40 反常积分p157}

\section{考点41 平面图形的面积p161}

\section{考点42 空间图形的体积p165}

\section{考点43 平面曲线的弧长p168}

\section{考点44 旋转曲面的侧面积p169}

\section{考点45 定积分的物理应用p170}

\chapter{向量代数与空间解析几何}

\section{考点46 向量及其运算p173}

\section{考点47 平面积空间直线p175}

\section{考点48 曲面及空间曲线p180}

\chapter{多元函数微分学}

\section{考点49 二元函数的极限及连续p185}

\section{考点50 偏导数p188}

\section{考点51 全微分p192}

\section{考点52 复合函数的偏导数与全微分p195}

\section{考点53 隐函数的偏导数及全微分p200}

\section{考点54 极值与最值p203}

\section{考点55 多元函数微分学的几何应用p209}

\section{考点56 方向导数与梯度p212}

\section{考点57 二元函数的二阶泰勒公式p214}

\chapter{多元函数积分学}

\section{考点58 二重积分概念与几何意义p217}

\section{考点59 直角坐标计算二重积分p221}

\section{考点60 极坐标计算二重积分p224}

\section{考点61 换序及换系p227}

\section{考点62 需分区域计算的几种情况p233}

\section{考点63 先一后二法(投影法)与先二后一法(截面法)p236}

\section{考点64 利用球面坐标计算三重积分p240}

\section{考点65 三重积分的性质及换序p241}

\section{考点66 第一类(平面、空间)曲线积分p242}

\section{考点67 第二类平面曲线积分p246}

\section{考点68 第一类曲面积分p253}

\section{考点69 第二类曲面积分p256}

\section{考点70 第二类空间曲线积分的计算p262}

\section{考点71 多元函数积分学的应用及场论初步p264}

\chapter{无穷级数}

\section{考点72 用定义和基本性质判断技术的敛散性p270}

\section{考点73 正项级数敛散性的判别方法p272}

\section{考点74 交错基数敛散性的判别方法p276}

\section{考点75 任意项基数敛散性的判别方法p278}

\section{考点76 收敛发散的证明题p281}

\section{考点77 幂级数的收敛半径及收敛域的求法p282}

\section{考点78 求一般函数项级数的收敛域p286}

\section{考点79 函数展开成幂级数p286}

\section{考点80 幂级数的和函数的求法p290}

\section{考点81 常数项级数的求和p295}

\section{考点82 傅立叶级数p297}

\chapter{常微分方程}

\section{考点83 微分方程的基本概念p301}

\begin{enumerate}
    \item 微分方程的通解与特解\\
        不含任意常数的解称为特解\\
        含独立任意常数个数与微分方程阶数相同的解称为微分方程的通解\\
        阶数与个数相同
    \item 线性与非线性微分方程\\
    看与y有关的次方是几次,若大于等于2次即为非线性

\end{enumerate}

\section{考点84 一阶微分方程p302}

\section{考点85 二阶可降阶的微分方程p307}

\section{考点86 常系数线性微分方程及欧拉方程p308}

\section{考点87 已知方程的解反求方程及进一步研究方程的解p312}

\section{考点88 通过变形改造建立微分方程并求解p318}

\section{考点89 微分方程的应用p323}

